\textbf{}%%% Thesis Introduction --------------------------------------------------
\chapter{Introduction}\label{Introduction}

\ifpdf
    \graphicspath{{Introduction/IntroductionFigs/PNG/}{Introduction/IntroductionFigs/PDF/}{Introduction/IntroductionFigs/}}
\else
    \graphicspath{{Introduction/IntroductionFigs/EPS/}{Introduction/IntroductionFigs/}}
\fi

In this chapter, we will explain the problem we aim to tackle in this project and what are the challenges we anticipate. We also outline 
the main objectives of the projects and our contributions. 

\section{Motivation}
Social media have literally changed our lives in the recent years. Social media platforms such as Facebook, Twitter
and many more not only offer their users a way to connect and communicate with their friends, colleagues and
family but they have been used as a way to trigger and sustain revolutions and protests. Shirky in [20] argues that
social media have the power to sustain political uprisings while others believe that the power of social media is
overrated [14]. Nevertheless, there have been several examples of people using social media to oust dictators and
protest against regimes. This is exactly what happened during the Iranian elections in 2009 and Egypt's uprising in
the beginning of 2011. During these events people used social media and especially Twitter, to report news during
protests and influence others to participate.
The events that took place during the Egyptian uprising engendered an unprecedented flow of information and
ideas on Twitter. People who were watching their Twitter timelines could literally see the events unfolding 
before their own eyes. One wonders what would have happened if someone collected all this information and tried
to make sense of them. What if we could be able to detect events and describe them as they are happening?
These are the questions this project sets out
to answer, not in a science fiction way but following a structured and scientific methodology. We have collected
social media data from Egypt's uprising and we have tried to describe what happened. Using this case study as a 
starting point we developed a generic framework for detecting and describing events as they unfold.\\\\
In the last few years a growing community of researchers started using Twitter to conduct research on social networks 
and data mining. Recently, Lotan et al.[13] attempted to investigate the events that took place during the
Egyptian revolution by collecting and analysing a large amount of tweets. They have identified different types
of users (media organizations, bloggers, activists) in the dataset and they studied how each type was influencing
the other by looking at the information flow between them. Another research project studied the information
dissemination during the Iranian elections and offered important insight into the dynamics of information propagation 
that are special to Twitter [23]. Other studies have investigated event detection and summarisation during
the Egyptian revolution as well as the different type of actors participating in the diffusion of news through Twitter.\\\\
Mining and analysing Twitter data is not an easy task and research has revealed numerous challenges for researchers.
The biggest challenge is the enormous amount of data flowing on Twitter every second requiring careful filtering in
order to block unwanted tweets such as updates on someone's life or spam tweets. Additionally, an intrinsic problem
of analysing on-line data is that they are not always related to real world occurrences. For example, Twitter specific
memes such as \#musicmonday(users tweet their music preferences) or \#followfriday(someone suggesting to other
users someone else to follow) produce a massive amount of tweets during a certain time interval but they do not
correspond in real life events.
Therefore, the motivation behind this project is firstly to overcome these difficulties and also contribute to the
research community by providing robust and accurate tools for automatically describing and detecting events from Twitter data. 

\section{Objectives}

This project sets out to answer the question if event extraction from Twitter historical data is possible. For this purpose 
we aim to review the current state-of-the-art in this area and gather the most prominent solutions. The study of existing methods will
help us identify the core methodology we need to employ and most importantly help us avoid common pitfalls by pointing out the biggest challenges
of mining Twitter data.\\\\
Based on this background research we will then proceed with the design and implementation of a data mining toolset that will allow us 
to extract events from Twitter data. This toolset should be able to overcome the main difficulties pertaining to mining Twitter data. 
Event identification alone might not be enough, therefore we also aim to construct algorithms that will summarise the event in order to aid human understanding.\\\\
In order to ensure that we have built a robust and effective toolset for event extraction a thorough evaluation of this toolset will take place where the methods used will be 
compared against each other, according to how well they solve our challenges.\\\\ 
The objectives above will lead to complete data mining system able to detect events and help humans understand what happened. The final product of our work would then be 
a prototypical application using a graphical user interface to allow humans to explore events in historical datasets from Twitter. 

\section{Contributions}
In the process of developing the solutions for the problems described above we have come across several
challenges as well as some opportunities. We have tried to tackle all the difficulties and also exploit the 
opportunities we were given. Our efforts have led to several contributions and the following list summarises the main contributions of this project:
\begin{itemize}
 \item \textbf{Literature survey:} In order to be able to develop our solution we have conducted an extensive literature survey about event detection, event summarisation and Twitter user classification. 
 Our research was focused on these three areas in the context of social media and more specifically on Twitter content. The emergence of social networks has sparked the interest of the research community and 
 numerous solutions have been proposed over the last few years. We have gathered the most prominent solutions in the literature and we have commented on their applicability and 
 feasibility in our project.
 \item \textbf{A data mining toolset:} The solution of the event detection and summarisation problems led to the development of a data mining toolset which consists of 
 several sub-components. These components vary from data acquisition tools to clustering algorithms and text processing tools. These
 individual components can act independently to solve smaller tasks but most importantly they can be combined to solve the problems posed 
 by this project.  
 \item \textbf{Evaluation of the toolset sub-components:} The core component of our solution for the event detection problem is text clustering. The challenging aspect of our work is that we had to deal with social media content
 which offers some advantages but at the same time it poses significant challenges. After implementing several clustering algorithms we have conducted a thorough evaluation study to assess their individual performance in several aspects that are unique to social media documents. Additionally, the user classification components have been evaluated as well and our results can be used as a guidance to future projects in this area.      
 \item \textbf{Pythia - a web application for event detection and summarisation:}We have built a proof-of-concept web application which uses our algorithm to detect events and summarize them using Twitter data. The application provides a user friendly
 environment and data visualisations to allow the user to discover events in historic data.
 \item \textbf{Pythia is used to conduct a case study on Arab Spring:}We have collected and analysed a large number of Twitter content related to the Arab Spring and we have used this dataset to conduct a case study on real data using
 our algorithm and web application.  	 
\end{itemize}\vspace{15pt}
\section{Report Structure}
TODO: Complete this section when the rest of the report is done.
%%% ----------------------------------------------------------------------


%%% Local Variables: 
%%% mode: latex
%%% TeX-master: "../thesis"
%%% End: 

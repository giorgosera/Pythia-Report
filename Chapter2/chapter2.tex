\chapter{A data mining framework for event detection}
\ifpdf
    \graphicspath{{Chapter2/Chapter2Figs/PNG/}{Chapter2/Chapter2Figs/PDF/}{Chapter2/Chapter2Figs/}}
\else
    \graphicspath{{Chapter2/Chapter2Figs/EPS/}{Chapter2/Chapter2Figs/}}
\fi

In this chapter we will explain in detail the methodology we have followed to construct the 
data mining system which can extract events from Twitter data. The system comprises of a number
of individual components which when combined together construct a complete framework for 
event extraction. These components are explained in detail in the sections below and a prototypical
application which uses this framework will be presented in Chapter \ref{chap:Application}.\\
The first section of this chapter introduces the motivating work for the implementation of the system
and we take a closer look at the concepts and key ideas that are necessary to understand the process we employ.

\section{Methodology overview}
The main idea behind our methodology for event detction is that similar events will be described by tweets having similar
content. Therefore, the main task in our methodology is to cluster the tweets in different groups. Having done that
we can then further investigate these clusters and identify which ones are events and also extract useful information from them.
This procedure is described in detail in the rest of this chapter and Figure 3.1 depicts an overview of the system. 
Initially, the system receives a stream of tweets from the database and process and transforms them in a format that aids clustering. 
The next step in the pipeline is the actual clustering of the tweets in order to detect groups of tweets discussing the same topic.
Finally, the extracted clusters are processed in order to identify the events.

\begin{figure}[!htbp]
  \begin{center}
    \includegraphics[height=1in, width=6in]{system-overview}
    \caption{System overview - The event extraction system comprises of several independent components.}
    \label{SystemOverview}
  \end{center}
\end{figure}

\section{Data preprocessing}
Mining and analysing text corpora are two well studied problems but in our case we face a slightly
different problem. Social media content and especially tweets are very short (140 characters)
and usually contain slang phrases, abbreviations and irrelevant information. Therefore, is of
foremost importance to ensure that clever data preprocessing takes place in order to help us in
the subsequent tasks. Preprocessing is also required to transform the documents (tweets) in a form that
will allow us to perform clustering on them. Therefore,  we define 'preprocessing' as all tasks which take place before transforming
the documents into the vector space representation. Figure 3.2 show the sub-components of the preprocessing module.

\begin{figure}[!htbp]
  \begin{center}
    \includegraphics[height=1.5in, width=6in]{preprocessing}
    \caption{The preprocessing module - All the steps neccessary to convert raw documents to vector space representation.}
    \label{PreprocessingOverview}
  \end{center}
\end{figure} 

\subsection{Background}

\subsubsection{Inverted index}
An inverted index, also known as an inverted file, is a data structure central to text-based information retrieval. The name is derived from its 
purpose and design which is to map key-value pairs, where a key is a term in a document and the value is the list of documents that contain this term.
For example if we have two documents:\\
\emph{Document1}: The cat is on the tree. \\
\emph{Document2}: The cat sat on the mat. \\
then the inverted index will look like:

\begin{center}
\begin{tabular}{ |l | l| }
  \hline
  \textbf{Key} & \textbf{Value} \\ \hline
  the & \{Document1, Document2\} \\
  cat & \{Document1, Document2\} \\
  is & \{Document1\} \\
  on & \{Document1, Document2\} \\
  tree & \{Document1\} \\
  sat & \{Document2\} \\
  mat & \{Document2\} \\
  \hline
\end{tabular}
\end{center}

The main reason for using an index is to increase the speed and efficiency of searches of the document 
collection. In our system the inverted index is vital component since it allows us to construct
term-documment vectors easily and also filter terms and documents. For example, using our index we
can find the words that appear either too often or less frequently and filter them out. This is used
to reduce the dimensionality of our dataset by removing unneccesary words. Alternatively, we
can remove documents/tweets which contain keywords that appear too often or less frequently.

\subsubsection{Vector space representation}
We can represent each document in a dataset by a vector of identifiers. Usually, these identifiers are the distinct words in the document and 
the resulting vector is called term-frequency vector. If we combine all the vectors for the documents in our dataset we will end up with an 
\boldmath $m \times n$  \unboldmath  matrix \boldmath $A$ \unboldmath. Each of the $m$ documents in the document collection are assigned a row 
in the matrix, while each of the $n$ unique terms in each document are assigned a column in the matrix. A non-zero element $a_{ij}$ in \boldmath 
$A$ \unboldmath, indicates not only that term $j$ occurs in document $i$, but also the number of times the term appears in that document. Since the number of terms in a given 
document is typically far less than the number of terms in the entire document collection, A is usually very sparse.\\
For example in Table 3.1 we see that $Document1$ contains three instances of the word team, while football occurs five times. We can also infer that
the words ball and world are missing for the entire document, as indicated by the value of zero at those entries of the matrix.  

The main advantage of transforming the documents in the vector space is that we can define vector-space similarities between documents and therefore
we can apply clustering algorithms on the document collection. In section we provide a more detailed discussion on similarity metrics and the vector space 
representation is used in clustering documents.

\begin{table}[tbp]
\centering
\begin{tabular}{ l  l  l  l  l  l  l }
  \hline
  \textbf{Document} & \textbf{team} & \textbf{ball} & \textbf{football} & \textbf{countries} & \textbf{world} & \textbf{england} \\ \hline
  \emph{Document1} & 3 & 0 & 5 & 1 & 3 & 1 \\
  \emph{Document2} & 2 & 1 & 2 & 0 & 8 & 2\\
  \emph{Document3} & 1 & 5 & 3 & 0 & 1 & 3\\
  \emph{Document4} & 5 & 0 & 1 & 2 & 5 & 4\\
  \hline
\end{tabular}
\caption{Term-frequency vector representation of documents}
\label{termfrequencyTable}
\end{table}

\subsubsection{Assigning weights to terms with TF-IDF weighting}
Once a document is transformed in its term-frequency vector we can assign weights to each term in the vector. So far the term-frequency vectors treat 
all the terms as equal but this may not be the case. For example, a word that appears more frequently than others in a document could be considered as an 
important word. At the same time words that appear frequently in the corpus, such as 'a', 'the', 'and', are not very useful and their 
importance must be discounted. \\
The most common method to solve this problem is the TF-IDF weight (TF-IDF stands for term frequency-inverse document frequency) which quantifies the importance of a term in a document of a document collection. More specifically, the more a word occurs in a document, and the less it occurs in the rest of the coprus, the higher its TF-IDF weighting 
will be. Mathematically, TF-IDF is expressed as:\\
\begin{eqnarray}
tf-idf_{t,d} = tf_{t, d} \times idf_t
\end{eqnarray}

where $tf_{t, d}$ is the importance of term $t$ in document $d$ and $idf_t$ is the importance of term $t$ relative to the entire corpus. $tf_{t,d}$ is higher when the term occurs many times in the document and $idf_t$ is higher when it occurs rarely in the dataset. Therefore, the TF-IDF weighting for a term is very high if the term occurs frequently in a single document but very rarely in the entire corpus and it is low when the term either occurs rarely in a document or frequently in the entire corpus. TF-IDF is widely used to compare the similarity between documents and a common use case is for search queries where the similarity of a query $q$ with a document $d$ is calculated using TF-IDF, providing
a sorted list of the most relevant documents. 
\\
   
\subsubsection{Feature selection}
TODO: Discusss feature selection methods used in our system.



\section{Document clustering}


\subsection{Background}
Give background information about the task of document clustering and the individual algorithms used.

\subsection{Kmeans algorithm}

\subsection{DBSACN algorithm}

\subsection{Non-negative matrix factorisation algorithm}

\subsection{Online clustering algorithm}

\section{Post-clustering analysis}

\subsection{Background}

\subsection{Generating automatic document cluster summaries}

\subsection{Detecting sentiment, named entities and locations in documents}

\section{Twitter user classification}

\subsection{Background}

\subsection{Decision trees}

\subsection{Neural networks}

\section{Summary}

Show a simple diagram with a lot of tweets becoming vectors and then those becoming clusters and then some of them events.
This is the outout of the algorithm.

% ------------------------------------------------------------------------

%%% Local Variables: 
%%% mode: latex
%%% TeX-master: "../thesis"
%%% End: 

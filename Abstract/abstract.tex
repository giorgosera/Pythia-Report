
% Thesis Abstract -----------------------------------------------------


%\begin{abstractslong}    %uncommenting this line, gives a different abstract heading
\begin{abstracts}        %this creates the heading for the abstract page

Social media platforms such as Facebook and Twitter not only offer their users a way to connect and communicate with their friends and
family but they have been used to trigger and sustain revolutions and protests. For example, during the Arab Spring 
journalists and common people were using Twitter to follow and explore the events that 
took place. Using this need for information as our motivation, we investigate how the use of text mining and machine 
learning, based on social media content, can help us to detect and explore real world events without human intevention. 
Mining social media corpora is a challenging task because we have to deal with a vast amount of 
documents which contain typos, spelling mistakes, abbreviations and they are very short. We propose a methodology, based on document 
clustering, which is able to identify events on Twitter. We take our work one step further by providing algorithms for automatic 
summarisation of an event in order to aid human understanding of the event. In the process of implementing our methodology we have developed a collection 
of data mining tools which when combined can extract and summarise events. In conjuction with this framework of tools, we 
present Pythia, a prototypical web application that is used to conduct a case study on a real dataset collected from 
Twitter during the Arab Spring. Finally, we present our evaluation of a number of different techniques and algorithms 
used in this project with respect to mining social media documents.  


\end{abstracts}
%\end{abstractlongs}


% ----------------------------------------------------------------------


%%% Local Variables: 
%%% mode: latex
%%% TeX-master: "../thesis"
%%% End: 

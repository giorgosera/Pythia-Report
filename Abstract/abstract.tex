
% Thesis Abstract -----------------------------------------------------


%\begin{abstractslong}    %uncommenting this line, gives a different abstract heading
\begin{abstracts}        %this creates the heading for the abstract page

Social media have literally changed our lives in the recent years. Social media platforms such as Facebook, Twitter
and many more not only offer their users a way to connect and communicate with their friends, colleagues and
family but they have been used as a way to trigger and sustain revolutions and protests. During the Arab Spring 
for example, journalists, politicians and common people were using Twitter to follow and explore the events that 
took place. Using this need for information as our motivation we investigate how the use of text mining and machine 
learning, based on historical data, can help us to detect and explore real world events without human intevention. 
This is a challenging task since mining social media corpora is fundamentally different from mining traditional online 
documents because we have to deal with a vast amount of documents which contain slang language, spelling mistakes, abbreviations 
and they are very short. We propose a methodology, based on document clustering, which is able to identify events on 
Twitter. We take our methodology one step further by providing algorithms for automatic summarisation of an event in 
order to aid human understanding of the event. In the process of implementing our methodology we have developed a collection 
of data mining tools which when combined can extract and summarise events. In conjuction with this framework of tools, we 
present Pythia, a prototypical web application that is used to conduct a case study on a real dataset collected from 
Twitter during the Arab Spring in 2011. Finally, we present a thorough evaluation of a number of different techniques and algorithms 
used in this project with respect to mining social media documents. Our evaluation can be used as a guidance for future research projects 
in this area. 


\end{abstracts}
%\end{abstractlongs}


% ----------------------------------------------------------------------


%%% Local Variables: 
%%% mode: latex
%%% TeX-master: "../thesis"
%%% End: 
